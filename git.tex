\documentclass{refcard}

\title{git}
\logo{git_logo.png}
\theme{default}

\begin{document}

\maketitle

\section{WHAT IS GIT?}

Git is a free and open source distributed version control system designed to handle everything from small to very large projects with speed and efficiency.\\

Git was created by Linus Torvalds in 2005 for development of the Linux kernel, with other kernel developers contributing to its initial development.\\

% --------------------------------------------------
% DEFINITIONS
% --------------------------------------------------
\section{some keywords}

\textbullet\ \textbf{repository:} where the entire snapshot of a project is stored\\
\textbullet\ \textbf{working directory:} local working copy of a project\\
\textbullet\ \textbf{commit:} latest snapshot of a project\\
\textbullet\ \textbf{branch:} parallel line of development\\
\textbullet\ \textbf{tag:} reference to a specific commit\\
\textbullet\ \textbf{fetch:} download and copy branch’s files to the working directory\\
\textbullet\ \textbf{push:} update a remote branch with the commits made to the current branch\\
%Git is an Open Source Distributed Version Control System originally developed by Linus Torvald.

% --------------------------------------------------
% CONFIGURE
% --------------------------------------------------
\section{configure}

% >> user name and mail
Configure user name and email adress:
\begin{ttyenv}
git config --global user.name "user_name"
git config --global user.email "user@email"
\end{ttyenv}

% >> editor
Configure default editor (nano, vim, notepad++, etc.)
\begin{ttyenv}
git config --global core.editor <editor>
\end{ttyenv}

% >> autocorrect
Set up auto-correct (the integer is the delay after which the suggested command will be executed)
\begin{ttyenv}
git config --global help.autocorrect 5
\end{ttyenv}

% >> output color
Set up terminal output color
\begin{ttyenv}
git config --global color.ui auto
\end{ttyenv}

% >> Win/Linux file format
%Set up automatic conversion of CRLF endlines to LF endlines
%\begin{ttyenv}
%git config --global core.autocrlf input
%\end{ttyenv}

% >> List
List all variables set in config file, along with their values
\begin{ttyenv}
git config --global --list
\end{ttyenv}

% >> Remove
Remove the line matching the key (\textit{i.e.} user.name) from config file
\begin{ttyenv}
git config --global --unset [key]
\end{ttyenv}

% --------------------------------------------------
% CREATE & GET
% --------------------------------------------------
\section{create \& get}

Initialize an existing \textbf{working directory} as a \textbf{remote repository}
\begin{ttyenv}
git init
\end{ttyenv}

Clone a project from a \textbf{remote repository}
\begin{ttyenv}
git clone [git_repo_url]
\end{ttyenv}
 

% --------------------------------------------------
% CHANGE
% --------------------------------------------------

\section{change}

Show the status of the \textbf{working directory} (tracked and untracked files)
\begin{ttyenv}
git status
\end{ttyenv}

Stage changes in files
\begin{ttyenv}
git add [files]
\end{ttyenv}

Unstage files but keep files modifications
\begin{ttyenv}
git reset <files>
\end{ttyenv}

\textbf{Commit} staged files
\begin{ttyenv}
git commit -m "commit_message"
\end{ttyenv}

\textbf{Commit} staged and tracked files
\begin{ttyenv}
git commit -am "commit_message"
\end{ttyenv}

Change an existing file path and stage the move
\begin{ttyenv}
git mv <source> <destination>
\end{ttyenv}

Delete a file from project and stage the removal for \textbf{commit}
\begin{ttyenv}
git rm <files>
\end{ttyenv}

Reset everything to the latest \textbf{commit}
\begin{ttyenv}
git reset --hard
\end{ttyenv}

Create a new \textbf{commit} and revert changes from the specified \textbf{commit}
\begin{ttyenv}
git revert [commit sha]
\end{ttyenv}

Remove untracked files from the \textbf{working directory} (\verb|-n| for dry-run)
\begin{ttyenv}
git clean
\end{ttyenv}

% --------------------------------------------------
% STASH
% --------------------------------------------------
\section{stash}

Save changes and return to the previous \textbf{commit} status
\begin{ttyenv}
git stash
\end{ttyenv}

List the stored stash
\begin{ttyenv}
git stash list
\end{ttyenv}

Apply back the saved stash
\begin{ttyenv}
git stash apply
\end{ttyenv}

Same as apply but remove the stash from the list
\begin{ttyenv}
git stash pop
\end{ttyenv}

Discard the changes from top of stash stack
\begin{ttyenv}
git stash drop
\end{ttyenv}

% --------------------------------------------------
% BRANCH
% --------------------------------------------------
\section{branch}

Create a \textbf{branch} (locally)
\begin{ttyenv}
git branch [branch_name]
\end{ttyenv}

Switch to a \textbf{branch}
\begin{ttyenv}
git checkout [branch_name]
\end{ttyenv}

Rename a \textbf{branch}
\begin{ttyenv}
git branch -m [new_branch_name]
\end{ttyenv}

Merge branchB into branchA 
\begin{ttyenv}
git checkout [branchA]
git merge [branchB]
\end{ttyenv}

Recover deleted \textbf{commit}
\begin{ttyenv}
git branch [branch_name] [commit]
\end{ttyenv}

Delete a \textbf{branch}
\begin{ttyenv}
git branch -d [branch_name]
\end{ttyenv}

% --------------------------------------------------
% INSPECT
% --------------------------------------------------
\section{inspect}

List \textbf{commit} history of current branch for the last n commits
\begin{ttyenv}
git log [-n count]
\end{ttyenv}

Show the \textbf{commit} history for the current \textbf{branch}
\begin{ttyenv}
git log [--graph --oneline --all --decorate]
\end{ttyenv}

Show the \textbf{commits} on branchA that are not on branchB
\begin{ttyenv}
git log branchA..branchB
\end{ttyenv}

Show the changes to files not yet staged
\begin{ttyenv}
git diff
\end{ttyenv}

Show the changes to staged files
\begin{ttyenv}
git diff --cached
\end{ttyenv}

Show all staged and unstaged file changes
\begin{ttyenv}
git diff HEAD
\end{ttyenv}

Show the changes between two \textbf{commit}
\begin{ttyenv}
git diff commit1 commit2
\end{ttyenv}

Show the diff of what is in branchA but not in branchB
\begin{ttyenv}
git diff branchA..branchB
\end{ttyenv}

% --------------------------------------------------
% TAG
% --------------------------------------------------
\section{tag}

List all \textbf{tags}
\begin{ttyenv}
git tag
\end{ttyenv}

Create a \textbf{tag} object named name for current \textbf{commit}
\begin{ttyenv}
git tag -a [tag_name]
\end{ttyenv}

Create a \textbf{tag} object named name for a specific \textbf{commit}
\begin{ttyenv}
git tag -a [tag_name] [commit_sha]
\end{ttyenv}

Delete a \textbf{tag} from \textbf{working directory}
\begin{ttyenv}
git tag -d [tag_name]
\end{ttyenv}


% --------------------------------------------------
% SYNCHRONIZE
% --------------------------------------------------
\section{synchronize}

\textbf{Fetch} changes from the \textbf{remote repository}, but not update tracking \textbf{branches}
\begin{ttyenv}
git fetch [remote]
\end{ttyenv}

\textbf{Fetch} changes from the \textbf{remote repository} and merge current \textbf{branch} with its upstream
\begin{ttyenv}
git pull [remote]
\end{ttyenv}

Transmit local \textbf{branch commits} to the \textbf{remote repository}, use \textbf{tags} option to push \textbf{tags}
\begin{ttyenv}
git push [--tags] [remote]
\end{ttyenv}

Push local \textbf{branch} to \textbf{remote repository}
\begin{ttyenv}
git push -u [remote] [branch]
\end{ttyenv}

\rflicense

\end{document}
